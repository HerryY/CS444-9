\documentclass[letterpaper,10pt]{article}

\usepackage{graphicx}
\usepackage{amssymb}
\usepackage{amsmath}
\usepackage{amsthm}

\usepackage{alltt}
\usepackage{float}
\usepackage{color}
\usepackage{url}

\usepackage{balance}
\usepackage[TABBOTCAP, tight]{subfigure}
\usepackage{enumitem}
\usepackage{pstricks, pst-node}

\usepackage{geometry}
\geometry{textheight=8.5in, textwidth=6in}

%random comment

\newcommand{\cred}[1]{{\color{red}#1}}
\newcommand{\cblue}[1]{{\color{blue}#1}}

\newcommand{\toc}{\tableofcontents}

%\usepackage{hyperref}

\def\name{McKenna Jones}

\title{Week Four Summary}
\author{McKenna Jones}
\date{April 24th, 2016}


%% The following metadata will show up in the PDF properties
% \hypersetup{
%   colorlinks = false,
%   urlcolor = black,
%   pdfauthor = {\name},
%   pdfkeywords = {cs311 ``operating systems'' files filesystem I/O},
%   pdftitle = {CS 311 Project 1: UNIX File I/O},
%   pdfsubject = {CS 311 Project 1},
%   pdfpagemode = UseNone
% }

\parindent = 0.0 in
\parskip = 0.1 in

\begin{document}

\begin{titlepage}
\maketitle
\end{titlepage}

\section{Chapter Six and Seven Summary}
Robert Love, the author of Linux Kernel Development (July 2, 2010), explains that most of the Linux Kernel is implemented using generic data structures, and that interrupts are necessary for hardware to communicate with the operating systems. To explain the role of data structures and interrupts in the Linux Kernel, the author discusses linked lists, queues, maps and binary trees, and the implementation of interrupts in the Linux Kernel, respectively. The purpose of these chapters is to inform the reader about which data structures are used in the Linux Kernel, and how interrupts are implemented. Based on the technical writing style of the author, the intended audience is anyone who has a basic knowledge of the Linux Kernel and would like to know more about data structures and interrupts in the Linux Kernel.

\end{document}
