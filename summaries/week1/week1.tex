\documentclass[letterpaper,10pt]{article}

\usepackage{graphicx}
\usepackage{amssymb}
\usepackage{amsmath}
\usepackage{amsthm}

\usepackage{alltt}
\usepackage{float}
\usepackage{color}
\usepackage{url}

\usepackage{balance}
\usepackage[TABBOTCAP, tight]{subfigure}
\usepackage{enumitem}
\usepackage{pstricks, pst-node}

\usepackage{geometry}
\geometry{textheight=8.5in, textwidth=6in}

%random comment

\newcommand{\cred}[1]{{\color{red}#1}}
\newcommand{\cblue}[1]{{\color{blue}#1}}

\newcommand{\toc}{\tableofcontents}

%\usepackage{hyperref}

\def\name{D. Kevin McGrath}

\title{Week One Summary}
\author{McKenna Jones}
\date{April 10th, 2016}


%% The following metadata will show up in the PDF properties
% \hypersetup{
%   colorlinks = false,
%   urlcolor = black,
%   pdfauthor = {\name},
%   pdfkeywords = {cs311 ``operating systems'' files filesystem I/O},
%   pdftitle = {CS 311 Project 1: UNIX File I/O},
%   pdfsubject = {CS 311 Project 1},
%   pdfpagemode = UseNone
% }

\parindent = 0.0 in
\parskip = 0.1 in

\begin{document}

\begin{titlepage}
\maketitle
\end{titlepage}

\section{Chapter 1 and 2 summary}
Robert Love, the author of Linux Kernel Development (July 2, 2010), explains the rich history of the Linux Kernel, which has deep roots in Unix, and the basics of the Linux Kernel, which is unique from any other large software project. The author does this by comparing the Linux Kernel to other Unix based kernels, and highlighting how to obtain, build, and become familiar with the Linux Kernel. The purpose of these first two chapters is to provide an introduction to the Linux Kernel in order to ensure that the reader has good background knowledge on the topic before they dive deeper into the Linux Kernel. The intended audience for these first two chapters is anyone who is interested about the Linux Kernel and eager to learn more about it, which is established by Love's in depth writing style.

\end{document}
