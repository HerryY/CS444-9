\documentclass[letterpaper,10pt]{article}

\usepackage{graphicx}
\usepackage{amssymb}
\usepackage{amsmath}
\usepackage{amsthm}

\usepackage{alltt}
\usepackage{float}
\usepackage{color}
\usepackage{url}

\usepackage{balance}
\usepackage[TABBOTCAP, tight]{subfigure}
\usepackage{enumitem}
\usepackage{pstricks, pst-node}

\usepackage{geometry}
\geometry{textheight=8.5in, textwidth=6in}

%random comment

\newcommand{\cred}[1]{{\color{red}#1}}
\newcommand{\cblue}[1]{{\color{blue}#1}}

\newcommand{\toc}{\tableofcontents}

%\usepackage{hyperref}

\def\name{McKenna Jones}

\title{Week Eight Summary}
\author{McKenna Jones}
\date{May 22nd, 2016}


%% The following metadata will show up in the PDF properties
% \hypersetup{
%   colorlinks = false,
%   urlcolor = black,
%   pdfauthor = {\name},
%   pdfkeywords = {cs311 ``operating systems'' files filesystem I/O},
%   pdftitle = {CS 311 Project 1: UNIX File I/O},
%   pdfsubject = {CS 311 Project 1},
%   pdfpagemode = UseNone
% }

\parindent = 0.0 in
\parskip = 0.1 in

\begin{document}

\begin{titlepage}
\maketitle
\end{titlepage}

\section{Chapter 11 and 16 Summary}
Robert Love, the author of Linux Kernel Development (July 2, 2010), explains that time is essential to the kernel, and that the kernel uses the disk cache to minimize disk I/O. As support for these claims regarding timers and the page cahce, the author explores absolute time, system time, dynamic time and temporal locality along with the speed of the page cache, respectively. The purpose of these chapters is to inform the reader on how timers and the page cache are managed in the Linux Kernel. Based on the technical writing style of the author, the intended audience is anyone who has a basic knowledge of the Linux Kernel and would like to know more about timers and the page cache in the Linux Kernel

\end{document}
