\documentclass[letterpaper,10pt]{article}

\usepackage{graphicx}
\usepackage{amssymb}
\usepackage{amsmath}
\usepackage{amsthm}

\usepackage{alltt}
\usepackage{float}
\usepackage{color}
\usepackage{url}

\usepackage{balance}
\usepackage[TABBOTCAP, tight]{subfigure}
\usepackage{enumitem}
\usepackage{pstricks, pst-node}

\usepackage{geometry}
\geometry{textheight=8.5in, textwidth=6in}

%random comment

\newcommand{\cred}[1]{{\color{red}#1}}
\newcommand{\cblue}[1]{{\color{blue}#1}}

\newcommand{\toc}{\tableofcontents}

%\usepackage{hyperref}

\def\name{D. Kevin McGrath}

\title{Week Two Summary}
\author{McKenna Jones}
\date{April 10th, 2016}


%% The following metadata will show up in the PDF properties
% \hypersetup{
%   colorlinks = false,
%   urlcolor = black,
%   pdfauthor = {\name},
%   pdfkeywords = {cs311 ``operating systems'' files filesystem I/O},
%   pdftitle = {CS 311 Project 1: UNIX File I/O},
%   pdfsubject = {CS 311 Project 1},
%   pdfpagemode = UseNone
% }

\parindent = 0.0 in
\parskip = 0.1 in

\begin{document}

\begin{titlepage}
\maketitle
\end{titlepage}

\section{Chapter 3 and 4 summary}
Robert Love, the author of Linux Kernel Development (July 2, 2010), explains that the concepts of process management and process scheduling are both essential abstractions of the Linux Kernel. To explain these concepts the author goes over how a process is defined, created, terminated and how the kernel decides which processes run, when they run, and for how long they run. The purpose of these two chapters is to first introduce the concept of a process and then define how they are used within the Linux Kernel in order to further the readers knowledge of the Linux Kernel. The intended audience is anyone who has a basic knowledge of the Linux Kernel and would like to know about the role of processes in the kernel, based on the technical writing style of the author. 



\end{document}
