\documentclass[letterpaper,10pt]{article}

\usepackage{graphicx}
\usepackage{amssymb}
\usepackage{amsmath}
\usepackage{amsthm}

\usepackage{alltt}
\usepackage{float}
\usepackage{color}
\usepackage{url}

\usepackage{balance}
\usepackage[TABBOTCAP, tight]{subfigure}
\usepackage{enumitem}
\usepackage{pstricks, pst-node}

\usepackage{geometry}
\geometry{textheight=8.5in, textwidth=6in}

%random comment

\newcommand{\cred}[1]{{\color{red}#1}}
\newcommand{\cblue}[1]{{\color{blue}#1}}

\newcommand{\toc}{\tableofcontents}

%\usepackage{hyperref}

\def\name{McKenna Jones}

\title{Week Five Summary}
\author{McKenna Jones}
\date{May 1st, 2016}


%% The following metadata will show up in the PDF properties
% \hypersetup{
%   colorlinks = false,
%   urlcolor = black,
%   pdfauthor = {\name},
%   pdfkeywords = {cs311 ``operating systems'' files filesystem I/O},
%   pdftitle = {CS 311 Project 1: UNIX File I/O},
%   pdfsubject = {CS 311 Project 1},
%   pdfpagemode = UseNone
% }

\parindent = 0.0 in
\parskip = 0.1 in

\begin{document}

\begin{titlepage}
\maketitle
\end{titlepage}

\section{Chapter Eight and Twelve Summary}
Robert Love, the author of Linux Kernel Development (July 2, 2010), explains that work in the Linux kernel is deferred three different ways, and that there a specific mechanisms used to manage memory in the Linux kernel. The concepts of softirqs, taskelts and work queues were used to explain how work is deferred, and bytes, pages, and zones were used to explain memory management. The purpose of these chapters is to inform the reader of the Linux implementations of bottom halves and memory management. Based on the technical writing style of the author, the intended audience is anyone who has a basic knowledge of the Linux Kernel and would like to know more about bottom halves and memory management in the Linux Kernel.

\end{document}
