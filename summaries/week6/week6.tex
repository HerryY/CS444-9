\documentclass[letterpaper,10pt]{article}

\usepackage{graphicx}
\usepackage{amssymb}
\usepackage{amsmath}
\usepackage{amsthm}

\usepackage{alltt}
\usepackage{float}
\usepackage{color}
\usepackage{url}

\usepackage{balance}
\usepackage[TABBOTCAP, tight]{subfigure}
\usepackage{enumitem}
\usepackage{pstricks, pst-node}

\usepackage{geometry}
\geometry{textheight=8.5in, textwidth=6in}

%random comment

\newcommand{\cred}[1]{{\color{red}#1}}
\newcommand{\cblue}[1]{{\color{blue}#1}}

\newcommand{\toc}{\tableofcontents}

%\usepackage{hyperref}

\def\name{McKenna Jones}

\title{Week Six Summary}
\author{McKenna Jones}
\date{May 8th, 2016}


%% The following metadata will show up in the PDF properties
% \hypersetup{
%   colorlinks = false,
%   urlcolor = black,
%   pdfauthor = {\name},
%   pdfkeywords = {cs311 ``operating systems'' files filesystem I/O},
%   pdftitle = {CS 311 Project 1: UNIX File I/O},
%   pdfsubject = {CS 311 Project 1},
%   pdfpagemode = UseNone
% }

\parindent = 0.0 in
\parskip = 0.1 in

\begin{document}

\begin{titlepage}
\maketitle
\end{titlepage}

\section{Chapter 15 and 17 Summary}
Robert Love, the author of Linux Kernel Development (July 2, 2010), explains that each process relies on the abstraction of virtual memory, and that the kernel has many functionalities which are used to implement device drivers. As support for these claims, the author explores how the kernel represents regions of memory, and how modules, kobjects and sysfs can be used to implement drivers. The purpose of these chapters is to inform the reader on how the process address space, and device drivers are managed in the Linux Kernel. Based on the technical writing style of the author, the intended audience is anyone who has a basic knowledge of the Linux Kernel and would like to know more the process address space and devices and modules in the Linux Kernel.

\end{document}
