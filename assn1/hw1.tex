\documentclass[letterpaper,10pt,titlepage,draftclsnofoot,onecolumn]{IEEEtran}

\usepackage{graphicx}                                        
\usepackage{amssymb}                                         
\usepackage{amsmath}                                         
\usepackage{amsthm}                                          

\usepackage{alltt}                                           
\usepackage{float}
\usepackage{color}
\usepackage{url}

\usepackage{balance}
\usepackage[TABBOTCAP, tight]{subfigure}
\usepackage{enumitem}
\usepackage{pstricks, pst-node}

\usepackage{geometry}
\geometry{textheight=8.5in, textwidth=6in}

%random comment

\newcommand{\cred}[1]{{\color{red}#1}}
\newcommand{\cblue}[1]{{\color{blue}#1}}

\usepackage{hyperref}
\usepackage{geometry}

\def\name{McKenna Jones}

\usepackage{titling}
\title{CS444 Project 1: Getting Acquainted}
\author{McKenna Jones}
\date{April 11th, 2016}

%pull in the necessary preamble matter for pygments output
\input{pygments.tex}

%% The following metadata will show up in the PDF properties
\hypersetup{
  colorlinks = true,
  urlcolor = black,
  pdfauthor = {\name},
  pdfkeywords = {cs444 ``operating systems'' },
  pdftitle = {CS 444 Project 1: Getting Acquainted},
  pdfsubject = {CS 311 Project 1},
  pdfpagemode = UseNone
}

\begin{document}
\begin{titlepage}


\maketitle
\begin{abstract}
test
\end{abstract}
\end{titlepage}
\section{Linux Kernel Setup}
\subsection{Commands used to setup kernel and qemu}
\begin{enumerate}
\item (os-class) mkdir CS444-064
\item (os-class) cd CS444-064
\item (os-class) git clone git://git.yoctoproject.org/linux-yocto-3.14
\item (os-class) git checkout v3.14.26
\item (os-class) source /scratch/opt/environment-setup-i586-poky-linux.csh
\item (os-class) cp /scratch/spring2015/files/config-3.14.26-yocto-qemu .config
\item (os-class) cp /scratch/spring2015/files/bzImage-qemux86.bin
\item (os-class) cp /scratch/spring2015/files/core-image-lsb-qemux86.ext3
\item (os-class) make -j4 all
\item (os-class) qemu-system-i386 -gdb tcp::5564 -S -nographic -kernel bzImage-qemux86.bin -drive file=core-image-lsb-sdk-qemux86.ext3,if=virtio -enable-kvm -net none -usb -localtime --no-reboot --append "root=/dev/vda rw console=ttyS0 debug"
\item (os-class) gdb
\item (gdb) target remote :5564
\item (gdb) continue
\end{enumerate}

\subsection{Explanation of flags for qemu command}
\begin{itemize}
\item -gdb tcp::5564 \\
Wait for gdb connection on device dev. In this case, dev is tcp::5564.
\item -S \\
Do not start CPU at startup
\item -nographic \\
This command totally disables graphical output so qemu is a simple command line application. 
\item -kernel bzImage-qemux86.bin \\
This uses bzImage as the kernel image.
\item -drive file=core-image-lsb-sdk-qemux86.ext3,if=virtio \\
This option defines which type of interface the drive is connected to. In this case it is a virtio drive. 
\item -enable-kvm \\
Enable KVM full virtualization support
\item -net none \\
This creates a new network interface card. For our purposes this is not nessesary
\item -usb \\ 
This enables the USB driver
\item -localtime \\ 
\item --no-reboot \\
\item --append "root=/dev/vda rw console=ttyS0 debug" \\
\end{itemize}

%input the pygmentized output of mt19937ar.c, using a (hopefully) unique name
%this file only exists at compile time. Feel free to change that.

\end{document}